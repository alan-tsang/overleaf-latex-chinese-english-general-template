\documentclass{template} %引用文档类

\usepackage{booktabs}  % 用于创建漂亮的表格,支持三线表等
\usepackage{setspace}  % 调整文档行距
\usepackage{amsmath}   % 提供数学公式排版支持
\usepackage{stfloats}  % 允许在双栏文档中使用跨页的浮动对象
\usepackage{graphicx}  % 支持插入图片
\usepackage{datetime}  % 提供日期和时间格式化
\usepackage{fancyhdr}  % 自定义页眉页脚 
\usepackage{caption}   % 改进浮动体标题的格式,提供\captionof命令
\usepackage{array}     % 提供更多的表格列格式选项
\usepackage{float}     % 提供浮动体排版控制
\usepackage{makecell}  % 允许表格中的单元格内换行
\usepackage{hyperref}  % 创建超链接,使文档可交互
\usepackage{multicol}  % 允许文档分为多栏
\usepackage{titlesec}  % 用于自定义章节标题的样式
\usepackage[ruled,linesnumbered]{algorithm2e}  % 引入算法环境宏包
\usepackage{ifthen}    %这个宏包提供逻辑判断命令
\usepackage{bm} %粗体数学符号的支持

% 自定义宏 等价于C++ define
\newcommand{\major}{Computer Science and Technology}
\newcommand{\name}{曾志存}
\newcommand{\stuid}{Alan Tsang}
\newcommand{\Name}{Zhicun Zeng} %宏命令,\Name->Name用于英文标题
\newcommand{\newtitle}{Latex模板}

%启用下面三行,可将图、表、参考文献的中文题注和标题变为英文
\renewcommand{\refname}{References}
\captionsetup[figure]{name=fig,labelsep=colon}
\captionsetup[table]{name=table,labelsep=colon}


%设置主要字体为TimesNew Roman
\setmainfont{Times New Roman} 
% 为章节标题定义新字体 Times New Roman
\newfontfamily\sectionef{Times New Roman} 
% 为中文章节标题定义新的楷书字体
\newCJKfontfamily\sectioncf{kaishu}


% 自定义一级标题的格式
\titleformat{\section}{   
  % 左对齐
  \raggedright           
  % 字体大小为正常大小,加粗
  \normalsize\bfseries
}{}{-1em}{}  % 间距调整,负值表示标题与编号的距离   

% 自定义二级标题格式
\titleformat*{\subsection}{\raggedright\small\bfseries}
% 自定义三级标题格式
\titleformat*{\subsubsection}{\raggedright\small\sectioncf}

% 重新定义 section 编号格式为阿拉伯数字
% \thesection表示当前 section 的编号格式
% \arabic表示阿拉伯数字
\renewcommand\thesection{\arabic{section}}

\setlength{\parindent}{2em}  % 设置段落首行缩进为 2em
\lstset{language=Matlab} % 设置代码块语言为 Matlab

\newboolean{first} % 引入布尔变量
\setboolean{first}{true} % 将布尔变量设置为true
\captionsetup{font={small}} % 设置图表标题的字体为小号

\pagestyle{fancy} % 使用fancy风格的页眉页脚

% 定义左中右页眉
\lhead{曾志存 / Alan Tsang}
\chead{GPL许可证}
\rhead{\thepage}

% 一栏图模板
% \begin{figure}[H]
%     \centering
%     \includegraphics[width=\linewidth]{name.png}
%     \caption{引用名}
%     \label{图注}
% \end{figure}

% 两栏图,定格在下一页
% \begin{figure*}[t]
%     \centering
%     \includegraphics[width=1.0\linewidth]{pipeline.png}
%     \caption{引用名}
%     \label{图注}
% \end{figure*}

% 一栏图模板,\resizebox可调整大小
% \begin{table}[H]
%   \centering
%   \caption{引用名}
%   \label{图表}
%   \small
%   \resizebox{0.5\textwidth}{!}{
%   \begin{tabular}{c|c|ccc}
%     \hline 
%     column1 & column2 & column3 & column4 & column5 & \\
%     \hline 
%     column1 & column2 & column3 & column4 & column5 & \\
%     column1 & column2 & column3 & column4 & column5 & \\
%     column1 & column2 & column3 & column4 & column5 & \\
%     column1 & column2 & column3 & column4 & column5 & \\
%     \hline 
%   \end{tabular}}
% \end{table}


\begin{document}

% 报告的标题与基本信息
\begin{center}
\zihao{-2} \textbf{\newtitle}\\
\zihao{4} \kaishu \name \ \ (\stuid)\\
\zihao{5} \kaishu 2024年1月24日\\中南大学计算机科学与工程学院,湖南省 长沙市 410083\\
\end{center} 

\zihao{-5}\textbf{摘\quad 要:}
\setstretch{1.3}此模板来源于个人对overleaf模板的改写,特点如下:1.元素丰富:包含基础的双栏图、单栏图、双栏表、单栏表、单栏算法的模板和使用例子;2.同时支持(纯)中英文;3.以bib格式引用参考文献;4.使用单栏的时候可能存在上下文空白较多的问题,建议多写点字填满,暂未找到好的解决方法(欢迎PR)。

\zihao{-5}\textbf{关键词:}latex模板;latex模板;latex模板;
~\\

\begin{center}
	\zihao{3} \textbf{Latex Template}\\
	\zihao{5} \Name\quad (\stuid)\\
	\zihao{5} January 24, 2024 \\ School of Computer Science and Engineering, Central South University, Changsha 410083, China
\end{center}

\zihao{5}\textbf{Abstract:}
\setstretch{1.3}this is a demo text;this is a demo text;this is a demo text;this is a demo text;this is a demo text;this is a demo text;this is a demo text;this is a demo text;this is a demo text;this is a demo text;this is a demo text;this is a demo text;this is a demo text;this is a demo text;

\zihao{5}\textbf{Keywords: }latex template;latex template;latex template;

% \begin{multicols}{num} 
% 用于实现多栏排版。num 参数表示希望分为多少列,可以是任意正整数。
\begin{multicols}{2}

    \section{引言 Iroduction}
这是一段latex示例文字\cite{reddy2013polypharmacology},这是一段latex示例文字,这是一段latex示例文字,这是一段latex示例文字,这是一段latex示例文字,这是一段latex示例文字,这是一段latex示例文字,这是一段latex示例文字,这是一段latex示例文字,这是一段latex示例文字,这是一段latex示例文字,这是一段latex示例文字,这是一段latex示例文字,这是一段latex示例文字,这是一段latex示例文字,这是一段latex示例文字,这是一段latex示例文字,这是一段latex示例文字,这是一段latex示例文字,这是一段latex示例文字,这是一段latex示例文字,这是一段latex示例文字,这是一段latex示例文字,这是一段latex示例文字,这是一段latex示例文字,这是一段latex示例文字,这是一段latex示例文字。



这是一段latex示例文字\cite{li2020learn},这是一段latex示例文字,这是一段latex示例文字,这是一段latex示例文字,这是一段latex示例文字,这是一段latex示例文字,这是一段latex示例文字,这是一段latex示例文字,这是一段latex示例文字,这是一段latex示例文字,这是一段latex示例文字,这是一段latex示例文字,这是一段latex示例文字,这是一段latex示例文字,这是一段latex示例文字,这是一段latex示例文字,这是一段latex示例文字,这是一段latex示例文字,这是一段latex示例文字,这是一段latex示例文字,这是一段latex示例文字,这是一段latex示例文字,这是一段latex示例文字,这是一段latex示例文字,这是一段latex示例文字,这是一段latex示例文字,这是一段latex示例文字。


    \section{第二部分}
这是一段latex示例文字\cite{reddy2013polypharmacology},这是一段latex示例文字,这是一段latex示例文字,这是一段latex示例文字,这是一段latex示例文字,这是一段latex示例文字,这是一段latex示例文字。

这是一段latex示例文字\cite{li2020learn},这是一段latex示例文字,这是一段latex示例文字,这是一段latex示例文字,这是一段latex示例文字,这是一段latex示例文字,这是一段latex示例文字。
    
        \subsection{这是二级标题}
        \begin{itemize}
        \item 这是一段latex示例文字\cite{reddy2013polypharmacology},这是一段latex示例文字,这是一段latex示例文字,这是一段latex示例文字,这是一段latex示例文字,这是一段latex示例文字,这是一段latex示例文字,这是一段latex示例文字,这是一段latex示例文字,这是一段latex示例文字,这是一段latex示例文字,这是一段latex示例文字,这是一段latex示例文字,这是一段latex示例文字,这是一段latex示例文字,这是一段latex示例文字,这是一段latex示例文字,这是一段latex示例文字,这是一段latex示例文字,这是一段latex示例文字,这是一段latex示例文字,这是一段latex示例文字,这是一段latex示例文字,这是一段latex示例文字,这是一段latex示例文字,这是一段latex示例文字,这是一段latex示例文字。

        \item 这是一段latex示例文字\cite{reddy2013polypharmacology},这是一段latex示例文字,这是一段latex示例文字,这是一段latex示例文字,这是一段latex示例文字,这是一段latex示例文字,这是一段latex示例文字,这是一段latex示例文字,这是一段latex示例文字,这是一段latex示例文字,这是一段latex示例文字,这是一段latex示例文字,这是一段latex示例文字,这是一段latex示例文字,这是一段latex示例文字,这是一段latex示例文字,这是一段latex示例文字,这是一段latex示例文字,这是一段latex示例文字,这是一段latex示例文字,这是一段latex示例文字,这是一段latex示例文字,这是一段latex示例文字,这是一段latex示例文字,这是一段latex示例文字,这是一段latex示例文字,这是一段latex示例文字。
        \end{itemize} 

        \subsubsection{三级标题}
        
        
    \section{图、表、算法等模板}

\subsection{算法}
算法\ref{alg:algorithm1}。 


% 双栏目使用 [H] 可以强制将浮动体放置在代码所在位置,而不受浮动机制的调整影响。
% 手动添加算法上下的行间距,可能有改进方法
\vspace{0.5em}
\begin{algorithm}[H]
	\caption{算法框架}
	\label{alg:algorithm1}
	\KwIn{训练集: $p(\mathcal{T})$; 测试集: $\mathcal{T}_{mt}$;学习率: $\alpha_{2}$.}
	\KwOut{结果 $\mathcal{T}_{mt}$.}  
	\BlankLine
        随机初始化 $\bm{\theta}$;
        
	
	\While{\textnormal{不收敛}}{
		\color{red}从训练集中进行采样$\mathcal{T}_{i} \sim p(\mathcal{T})$; \color{black}
		
		\ForEach{$\mathcal{T}_{i}$中的任务}{
		      使用 $\mathcal{S}_{i}$计算        $\mathcal{L}_{\mathcal{T}_{i}}\left(f_{\bm{\theta}}\right)$ ;
		      计算学习后参数$\bm{{\theta}^{\prime}_{i}}$;
			
			使用 $\mathcal{Q}_{i}$计算$\mathcal{L}_{\mathcal{T}_{i}}\left(f_{\bm{{\theta}_{i}^{\prime}}}\right)$ ;
		}
		更新 $\bm{\theta}$; 
	}
	
	使用支持向量集 $\mathcal{T}_{mt}$,计算学习后的参数 $\bm{{\theta}^{\prime}_{mt}}$ ; 
	
	使用模型$f_{\bm{{\theta}_{mt}^{\prime}}}$来预测查询集$\mathcal{T}_{mt}$中元素的标签.

\end{algorithm}
% 手动添加算法上下的行间距,可能有改进方法
\vspace{0.5em}

    \begin{enumerate}
        \item 有序列表;有序列表;有序列表;有序列表;有序列表;有序列表;有序列表;有序列表;有序列表;有序列表;有序列表;有序列表;有序列表;有序列表;有序列表;
        \item 有序列表;有序列表;有序列表;有序列表;有序列表;有序列表;有序列表;有序列表;有序列表;有序列表;有序列表;有序列表;有序列表;有序列表;有序列表;
        \item 有序列表;有序列表;有序列表;有序列表;有序列表;有序列表;有序列表;有序列表;有序列表;有序列表;有序列表;有序列表;有序列表;有序列表;有序列表;

\subsection{数学公式}
公式\eqref{公式1}所示

公式\eqref{公式2}所示
\begin{equation}
\label{公式1}
    E_j(A, B)=\frac{A * B}{\|A\|^2+\|B\|^2-A * B}
\end{equation}
其中 $A, B$ 分别表示为两个向量, 集合中每个元素表示为向量中的一个维度, 在每个维度上, 取值通常是 $[0,1]$ 之间的值, $A\cdot B$表示向量乘积, $\|A\|^{\wedge} 2$ 表示向量的模, 即
\begin{equation}
\label{公式2}
    \|A\|^2=\sqrt{\sum_{i=1}^n A_i^2}
\end{equation}

    \end{enumerate}


\subsection{图}
图\ref{图注}
\begin{figure}[H]
    \centering
    \includegraphics[width=0.8\linewidth]{demo.jpg}
    \caption{引用名}
    \label{图注}
\end{figure}


% 插入跨栏图片时,结束双栏环境,否则figure、tabble都不生效
% 并且浮动环境的htbp全部不生效
% 跨栏表格同理
\end{multicols}


跨栏图\ref{图注2}
\begin{figure}[H]
    \centering
    \includegraphics[width=0.8\linewidth]{demo.jpg}
    \caption{引用名}
    \label{图注2}
\end{figure}

\begin{multicols}{2}

\subsection{代码块}
\begin{lstlisting}[language=Python, caption=示例代码, label=code:example]
def hello_world():
    print("Hello, world!")

# 
hello_world()
\end{lstlisting}

\subsection{表}
这是一段latex示例文字,这是一段latex示例文字,这是一段latex示例文字,这是一段latex示例文字,这是一段latex示例文字,这是一段latex示例文字,这是一段latex示例文字,这是一段latex示例文字,这是一段latex示例文字,这是一段latex示例文字,这是一段latex示例文字,这是一段latex示例文字,这是一段latex示例文字,这是一段latex示例文字。
表\ref{表注}
% 一栏图模板,\resizebox可调整大小
\begin{table}[H]
  \centering
  \caption{引用名}
  \label{表注}
  \small
  \resizebox{0.5\textwidth}{!}{
  \begin{tabular}{c|c|ccc}
      \hline 
      列1 & 列2 & 列3 & 列4 & 列5 \\
      \hline 
      数据1 & 数据2 & 数据3 & 数据4 & 数据5 \\
      数据1 & 数据2 & 数据3 & 数据4 & 数据5 \\
      数据1 & 数据2 & 数据3 & 数据4 & 数据5 \\
      数据1 & 数据2 & 数据3 & 数据4 & 数据5 \\
      数据1 & 数据2 & 数据3 & 数据4 & 数据5 \\
      \hline 
  \end{tabular}}
\end{table}
    
    % 引用
    % unsrt按照引用顺序排序
    \bibliographystyle{unsrt}
    \bibliography{demo.bib}
    
    \end{multicols}
    

\end{document}
